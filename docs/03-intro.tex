\section*{ВВЕДЕНИЕ}
\addcontentsline{toc}{section}{ВВЕДЕНИЕ}

\textbf{Тут точно нужно что-то поменять}

В последнее десятилетие центральное процессорное устройство  достигли своей пиковой тактовой частоты -- около 5 Ггц, и этот предел будет преодолён ещё не скоро \cite{cpu_pick}. Из-за того что ЦПУ стали настолько быстрыми, становится нередка ситуация, когда процессор не выполняет инструкции, а ждёт, пока данные переместятся с диска в оперативное запоминающее устройство (или наоборот) \cite{in-kernel-memory-compression}. Так, например, скорость работы системы с мощным ЦПУ, но с маленьким количеством ОЗУ может быть маленькой -- несмотря на быстроту, ЦПУ будет часто будет ожидать подсистему ввода/вывода \cite{in-kernel-memory-compression}.

Цель работы -- изучить метод сжатия страниц виртуальной памяти в оперативной памяти в ядре Linux.

Для достижения поставленной цели потребуется:

\begin{itemize}
	\item описать термины предметной области и обозначить проблему;
	\item дать характеристику архитектуры ядер операционных систем;
	\item изучить подходы, структуры данных и функции в ядре Linux, позволяющие управлять оперативной памятью;
	\item описать работу модуля сжатия оперативной памяти в ядре Linux.
\end{itemize}

\pagebreak
