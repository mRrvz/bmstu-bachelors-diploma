\section*{ЗАКЛЮЧЕНИЕ}
\addcontentsline{toc}{section}{ЗАКЛЮЧЕНИЕ}

\textbf{Это надо поменять}

Были рассмотрены понятия предметной области: сжатия данных, оперативной памяти и виртуальной памяти. 

Были характеризованы современные ядра операционных систем: с монолитной и микроядерной архитектурой, описаны их плюсы и минусы.

Проведён краткий обзор ядра Linux, структур данных и функций отвечающих за управление памятью внутри ядра: \texttt{struct page}, \texttt{alloc\_pages()}, \texttt{alloc\_page()}.

Описана работа модуля zRam, позволяющего хранить страницы виртуальной памяти в сжатом виде оперативной памяти. Рассмотрены основные структуры данных, использующиеся в этом модуле: \texttt{struct zram} и\\ \texttt{struct zram\_table\_entry}.

Таким образом, цель работы, изучить метод сжатия страниц виртуальной памяти в оперативной памяти в ядре Linux, была достигнута.

\pagebreak