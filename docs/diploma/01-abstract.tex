\section*{РЕФЕРАТ}

Расчетно-пояснительная записка \pageref{LastPage} с., \totalfigures\ рис., \totaltables\ табл., 37 ист.

Объектом исследования данной работы является сжатие страниц оперативной памяти в ядре Linux. Сжатие данных, коэффициент сжатия которых будет приближен к единице, практически бесполезно. Другими словами, попытки сжать данные, которые сжимаются плохо, или не сжимаются вообще, лишь тратят машинное время и практически не дают никакого преимущества. Целью этой работы является исследование метода сжатия страниц оперативной памяти в ядре Linux и разработка оптимизации для данного метода.

Для достижения поставленной цели необходимо решить следующие задачи:

\begin{itemize}[leftmargin=1.6\parindent]
	\item [---] рассмотреть методы увеличения количества оперативной памяти;
	\item [---] дать характеристику архитектурам ядер современных операционных систем;
	\item [---] изучить подходы, структуры данных и API \cite{api} в ядре Linux, позволяющие управлять подсистемой памяти;
	\item [---] описать работу модуля сжатия оперативной памяти в ядре Linux;
	\item [---] разработать оптимизацию для данного модуля;
	\item [---] спроектировать структуру программного обеспечения, реализующего оптимизацию модуля сжатия оперативной памяти;
	\item [---] сравнить метод сжатия страниц оперативной памяти с оптимизацией и без.
\end{itemize}

Поставленная цель достигнута: в ходе дипломной работы был разработана оптимизация метода сжатия страниц оперативной памяти в ядре Linux. Разработанная оптимизация повышает скорость работы метода сжатия страниц: это достигается благодаря незначительным жертвам в размере сжатых данных.

КЛЮЧЕВЫЕ СЛОВА

\textit{сжатие данных, оперативная память, linux, zram, операционные системы.}

\pagebreak