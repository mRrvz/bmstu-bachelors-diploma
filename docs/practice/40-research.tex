\section{Исследование разработанной оптимизации метода сжатия страниц оперативной памяти в ядре Linux}

В рамках дипломной работы было проведено исследование сравнения времени обработки и коэффициент сжатия файлов в блочном устройстве zram с оптимизацией и без. Результаты исследования представлены в данном разделе.

\subsection{Описание используемых данных}

Для исследования работоспособности разработанного программного обеспечения и оценки времени обработки данных и коэффициента их сжатия были выбраны бинарные файлы различного типа и разного размера. В исследовании использовались файлы типа pdf (portable document format \cite{pdf}), apk (android package \cite{apk}) и случайные файлы из домашней директории, сохраненные в единый файл с помощью утилиты tar \cite{tar}. Размер файлов составляет 400 мегабайт, 430 мегабайт и 5 гигабайт соответственно.

\subsection{Методика проведения исследования}

Для исследования необходимо произвести замеры количества машинных инструкций, времени сжатия и коэффициента сжатия данных с использованием энтропийной оптимизации и без. Количество машинных инструкций и время выполнения подсчитывается с помощью утилиты perf \cite{perf}, а коэффициент сжатия данных с использованием встроенной в модуль zram статистики.

В таблице 2 представлены результаты сравнения коэффициентов сжатия с включенной разработанной оптимизацией и без. В первом столбце указано, включена энтропийная оптимизация или нет. В ячейках таблицы указан исходный размер данных, сжатый и коэффициент сжатия.

\begin{table}[!htb]
	\label{table:coeffs}
	\begin{center}
		\caption{Таблица сравнения коэффициентов сжатия с оптимизаций и без}
		\begin{tabular}{|c|c|c|c|c|}
			\hline
			\bfseries Патч & \bfseries Файл & \bfseries Размер на входе, кб & \bfseries  Размер на выходе, кб & \bfseries Коэф. сжатия \\
			\hline
			Да & pdf & 397.464 & 396.535 & 1.002 \\ 
			Нет & pdf & 397.464 & 395.870 & 1.004 \\ \hline
			Да & apk & 421.340 & 327.992 & 1.284 \\ 
			Нет & apk & 421.340 &  282.331 & 1.492 \\ \hline
			Да & tar & 5.153.692 & 4.131.741 & 1.247 \\ \
			Нет & tar & 5.153.692 & 4.117.655 & 1.251 \\ \hline
		\end{tabular}
	\end{center}
\end{table}

В таблице 3 представлено сравнение времени выполнения и количества машинных инструкций с включенной оптимизацией и без.

\begin{table}[!htb]
	\label{table:time}
	\begin{center}
		\caption{Таблица сравнения количества машинных инструкций с оптимизаций и без}
		\begin{tabular}{|c|c|c|c|c|}
			\hline
			\bfseries Патч & \bfseries Файл & \bfseries Размер на входе, кб & \bfseries Время сжатия, с& \bfseries Кол-во инструкций \\
			\hline
			Да & pdf & 397.464 & 0.572 & 4.641.658.347 \\ 
			Нет & pdf & 397.464 & 2.098 & 17.187.405.320 \\ \hline
			Да & apk & 421.340 & 1.632 & 13.055.231.312 \\ 
			Нет & apk & 421.340 & 3.196 & 24.345.720.313 \\ \hline
			Да & tar & 5.153.692 & 7.066 & 35.039.839.769 \\ \
			Нет & tar & 5.153.692 & 11.955 & 76.763.123.464 \\ \hline
		\end{tabular}
	\end{center}
\end{table}

\subsection{Вывод}

В результате исследования было установлено что коэффициент сжатия сильно зависит от входных данных. Так, например, файл формата pdf практически не сжался как с включенной энтропийной оптимизацией, так и без.

Оптимизация метода сжатия ускоряет процесс преобразования данных в среднем от 2 до 4 раз. Чем меньше процент сжатия исходных данных без включенной оптимизации, тем больше ускоряется процесс сжатия с включенной оптимизацией. Файл формата pdf потерял маленькую долю сжатия (коэффициент сжатия стал 1.002, вместо 1.004), но при этом, процесс преобразования данных ускорился в 4 раза.

Разносортные данные упакованные в единый файл сжимаются в среднем на 25\%. Сжатие с энтропийной оптимизацией происходит в 1.9 раза быстрее, чем без. При этом, потеря в сжатии составляет менее процента.

Подводя итог, можно сделать вывод, что разработанная оптимизация программного обеспечения ускоряет процесс сжатия в несколько раз (2-4 раза), при этом потеря в сжатии не крайне малы: от 1\% до 14\%, в зависимости от входных данных. 

Полученные результаты, вместе с модификацией модуля zram, были отправлены в качестве RFC (англ. request for comments \cite{rfc}) письма мейнтейнерам модуля ядра zram \cite{rfc-patch-kernel}.

\pagebreak